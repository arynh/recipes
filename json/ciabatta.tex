\documentclass[12pt]{article}
\usepackage[margin=1in,landscape]{geometry}
\usepackage{multicol}
\author{Paul Hollywood}
\title{Ciabatta}
\date{}
\begin{document}
\begin{multicols*}{2}
\maketitle

Source: http://paulhollywood.com/recipes/ciabatta/

\section{Notes}

This straightforward ciabatta recipe is relatively easy and satisfying to make To get that classic ciabatta shape and open texture, you need a very wet and sloppy dough, so you really have to make it in an electric mixer. Serve this thin-crusted, light-textured bread warm for breakfast, with soups or salads, or split, toasted and filled with salami, prosciutto or cheese for an Italian-style sandwich.

\section{Ingredients}

\begin{itemize}
    \item 500g strong white bread flour, plus extra for dusting
    \item 10g salt
    \item 10g instant yeast
    \item fine semolina for dusting
\end{itemize}

\section{Method}

\begin{enumerate}
    \item Put the flour, salt and yeast with 330ml cold water into a freestanding mixer fitted with a dough hook (don’t put the salt directly on top of the yeast). Begin mixing on a slow speed.
    \item As the dough starts to come together, with the motor running, slowly add another 110ml of cold water, drip by drip. Mix for a further 5-8 minutes on a medium speed until the dough is smooth and stretchy.
    \item Lightly oil a 3 litre square plastic container with a lid. (It’s important to use a square tub as it helps shape the dough).
    \item Tip the dough into the oiled container and seal with the lid. Leave for 1.5 to 1.75 hours at room temperature, or until at least doubled, even trebled in size (it's important the dough proves slowly, otherwise it will collapse and your loaves will be flat).
    \item Dust two large baking trays with flour and semolina.
    \item Dust your work surface heavily with flour and semolina and carefully tip out the dough (it will be very wet); trying to retain a rough square shape.
    \item Rather than knocking it back, handle it gently so you can keep as much air in the dough as possible. Coat the top of the dough with more flour and/or semolina.
    \item Cut the dough lengthways, dividing into four equally-sized loaves. Stretch each piece of dough lengthways a little and place on the prepared baking trays. Leave the ciabatta to rest for a further 30-45 minutes.
    \item Preheat the oven to 220C/425F and bake for 25 minutes, or until the loaves are golden-brown and sound hollow when tapped on the base. Leave to cool completely on a wire rack before serving.
\end{enumerate}

\end{multicols*}
\end{document}
